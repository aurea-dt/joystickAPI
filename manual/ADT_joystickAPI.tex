\documentclass[a4paper,10pt]{article}
\usepackage{graphicx}
\usepackage{ucs}
\usepackage[utf8x]{inputenc} %instalar latex-ucs.
\usepackage[spanish]{babel}
\usepackage{listings}
\usepackage[colorlinks=true]{hyperref}

% Title Page
\title{ADT\_joystick API\\ Conexión con el puerto de juegos en C++}
\author{Mario Chririnos Colunga\\ Áurea - Desarrollo Tecnológico}


\begin{document}
\maketitle

% \begin{abstract}
% 
% \end{abstract}

\tableofcontents
\section{Introducción}
En este documento se describe nuestra interfaz de programación de aplicaciones para la conexión con \textit{joystick} o \textit{gamepads} conectados a un puerto de juegos. Se utlizia \href{http://library.gnome.org/devel/glib/}{glib} para monitorizar el estado del puerto de juegos y generar eventos. El documento está estructurado de la siguiente manera: en la sección \S\ref{api} describe el funcionamiento de nuestro código, la sección \S\ref{ejemplo} describe detalles relacionados al ejemplo incluido con el código fuente y la sección \S\ref{notas} proporciona notas y comentarios finales sobre este documento. 

\section{ADT\_joystick API}
\label{api}

\begin{figure}

  \lstset{language=c++}
  \lstset{commentstyle=\textit}
  \tiny
  \begin{lstlisting}[frame=trbl]{}
#ifndef ADT_JOYSTICK_H
#define ADT_JOYSTICK_H
//------------------------------------------------------------------------------
#include <glib.h>
#include <linux/joystick.h>
#include <vector>
using namespace std;
typedef void (*jsPt2Function)(unsigned int number, int value, void *userdata);
//------------------------------------------------------------------------------
class JsData
{
 public:
	char name[80];
	int nAxis;
	int nButtons;
	JsData();
};
//------------------------------------------------------------------------------
class ADT_joystick
{
 public:
	int enumJoysticks();
	int connect(const char *filename);
	int readJs();
	void setAxisCallback(jsPt2Function _fpt);
	void unsetAxisCallback();
	void setButtonCallback(jsPt2Function _fpt);
	void unsetButtonCallback();

	vector<JsData> getList() const;
	const char* getDevName() const;
	const char* getName() const;
	unsigned int getAxisCount() const;
	unsigned int getButtonCount() const;
	int getAxis(unsigned int i) const;
	bool getButton(unsigned int i) const;

	ADT_joystick();
	~ADT_joystick();
};
//------------------------------------------------------------------------------
#endif

  \end{lstlisting}
  \caption{ADT\_joystick.h}
  \label{joystick.h}
\end{figure}
  \subsection{Requisitos}
    Para compilar código de esta API se requiere contar con el paquete \textbf{libgtk2.0-dev}, agregar al compilador C++ las banderas obtenidas mediante: \texttt{pkg-config --cflags gtk+-2.0}  y contar con un nucleo de Linux versión 2.4 o superior.

  \subsection{Listado de dispositivos}
    Para en listar los puertos de juegos conectados al sistema se utiliza la función miembro \texttt{int enumJoysticks(void)} la cual devuelve el numero de dispositivos encontrados. La lista creada es del tipo \texttt{vector<JsData>}. La función getList() permita acceder a esta lista.
	
  \subsection{Conexión y desconexión}
    La función miembro \texttt{int connect(const char *filename)} permite conectarse el puerto de juegos especificado por \textit{filename}, e.g. \texttt{connect(/dev/input/js0)}. La función devuelve cero si la conexión fue exitosa.

    %Para desconectar el puerto se utiliza la función \texttt{void disconnect()}.

\subsection{Información del dispositivo}
Las siguientes funciones permiten obtener información acerca del puerto de juegos al que se esta conectado.
  \begin{itemize}
    \item getDevName() Regresa el nombre del puerto de juegos, e.g. \textit{/dev/input/js0}.
    \item getName() Regresa el nombre de identificación del puerto de juegos asignado por el fabricante.
    \item getAxisCount() Regresa el numero de ejes del puerto de juegos.
    \item getButtonCount() Regresa el numero de botones del puerto de juegos.
    \item getAxis(unsigned int i) Regresa el valor del eje \textit{i}.
    \item getButton(unsigned int i) Regresa el valor del botón \textit{i}.
  \end{itemize}

\subsection{Eventos}
La API utiliza \href{http://library.gnome.org/devel/glib/}{glib} para monitorizar el estado del puerto de juegos y llamar a una función de retro llamada (\textit{callback}) cuando se genere un evento. Se debe iniciar un \textit{gtk main loop} en el programa principal par 
activar los eventos.

Las funciones miembro \texttt{setAxisCallback(fpt)} y \texttt{setButtonCallback(fpt)} permiten conectar una retro llamada a un evento generado por un botón o un eje del puerto de juegos. La función a conectar debe de tener la firma \texttt{fpt(unsigned int number, int value, void *userdata)}. Las funciones miembro \texttt{unsetAxisCallback()} y\texttt{ unsetButtonCallback()} permiten desconectar las retro llamadas.

\section{Ejemplo}
\label{ejemplo}
  Junto con el código de esta API se provee de un programa ejemplo para demostrar su funcionamiento.

  \subsection{Requisitos}
  Para compilar el programa ejemplo se requiere contar con el paquete \textbf{libgtk2.0-dev}. Las banderas de compilación se especifican en el archivo \textit{makefile} de este ejemplo.

\section{Notas}
\label{notas}
El código fuente de esta API puede ser descargado en \href{http://www.aurea-dt.com/Software/joystickAPI.php}{nuestro sitio web}, en donde también se pueden reportar errores en el código fuente. Para reportar errores en este documento favor de escribir a errata@aurea-dt.com.

Para futuras versiones de esta API se contempla la calibración  y la retroalimentación háptica del joystick.



%\bibliographystyle{plain}	% (uses file "plain.bst")
%\bibliography{/home/mchc/Aurea/CPP/Programs/ADTlib-0.0.1/Manual/ADTlib.bib}
\end{document}          
